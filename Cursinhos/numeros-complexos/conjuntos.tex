\documentclass[a4paper, 12pt]{article}
\usepackage[top=2cm, bottom=2cm, left=1.7cm, right=1.7cm]{geometry}
\usepackage[utf8]{inputenc}
\usepackage{amsmath, amsfonts, amssymb}

\begin{document}

\begin{center}
\textbf{Liste de Números Complexos}
\end{center}

\textbf{Resumo}
\\


\begin{itemize}

\item $\mathbb{R} \longrightarrow$ \textbf{Reta} Real

\item $\mathbb{C} \longrightarrow$ \textbf{Plano} Complexo (também chamado de Plano de Argand-Gauss)

\item $ \mathbb{C} = \{ z = a + bi \, | \, a,b \, \in \mathbb{R} \textrm{, onde } i^2 = -1 \} $

\item A representação de um complexo como um ponto no plano recebe o nome de \textbf{afixo}

\item Multiplicar por $i$ corresponde a rotacionar um complexo exatamente $90^{\circ}$ no sentido anti-horário, mantendo o tamanho original do vetor. Para rotacionar por ângulos diferentes, bem como para mudar o módulo, multiplicaremos por números complexos específicos, dependendo de cada resultado que se queira obter.

\item Para converter de graus para radianos (ou vice-versa), faça a seguinte regra de três:

$\dfrac{\theta}{\alpha^{\circ}} = \dfrac{\pi}{180^{\circ}}$ , com $\theta$ em radianos

\item Forma algébrica (ou retangular)


	\begin{itemize}

\item $ \textrm{Parte Real e Parte Imagin\'aria: } a = \textrm{Re} (z) \textrm{ e } b = \textrm{Im} (z) \textrm{ representam as coordenadas do complexo no plano} $

\item $ \textrm{Igualdade: } z_1 = z_2 \longleftrightarrow a + bi = c + di \longrightarrow a = c \textrm{ e } b = d $

\item $ \textrm{Soma: } z_1 + z_2 = (a + bi) + (c + di) = (a + c) + (b + d)i \in \mathbb{C} $

\item $ \textrm{Multiplicação: } z_1 \cdot z_2 = (a + bi) \cdot (c + di) = (ac - bd) + (bc + ad)i \in \mathbb{C}$

\item $ \textrm{Conjugado: } z = a + bi \longrightarrow \overline{z} = a - bi $

\item $ \textrm{Divisão: } \dfrac{z_1}{z_2} = \dfrac{ z_1 \cdot \overline{z_2} }{z_2 \cdot \overline{z_2}} $

	\end{itemize}

\item Forma trigonométrica (ou polar)


	\begin{itemize}
	
\item Módulo: $ |z| $ é o tamanho do vetor que liga a origem até o afixo, no plano complexo	

\item Argumento: $ arg(z) = \theta $; ângulo entre a parte positiva do eixo dos reais, seguindo em sentido anti-horário, até o vetor que representa o número complexo
	
\item Notação: $ z = |z|cis(\theta) = (\, |z| ; arg(z) \,)$

\item Soma: \textbf{não faça somas nessa forma, porque é muito complicado e trabalhoso}

\item Multiplicação: $z_1 \cdot z_2 = |z_1| \cdot |z_2| \cdot cis(\theta_1 + \theta_2) $

\item Conjugado: $z = |z| \cdot cis(\theta) \longrightarrow \overline{z} = |z| \cdot cis (-\theta) $

\item Divisão: $ \dfrac{z_1}{z_2} = \dfrac{|z_1|}{|z_2|} \cdot cis(\theta_1 - \theta_2) $

	\end{itemize}


\item Conversão: algébrica $\longleftrightarrow$ trigonométrica


	\begin{itemize}
	
\item Algébrica $\longrightarrow$ trigonométrica

$ |z| = \sqrt{a^2 + b^2} $ e $ arg(z) = arctan (\dfrac{b}{a}) $

\item Trigonométrica $\longrightarrow$ algébrica

$ a = |z| \cdot cos(\theta) $ e $ b = |z| \cdot sin(\theta) $
	
	\end{itemize}


\item Fórmulas de De Moivre

	\begin{itemize}

\item As fórmulas de De Moivre falam de potenciação e radiciação com $n \in \mathbb{Z}$. Portanto, a menos que você fique confortável calculando coisas do tipo $z^n = (a + bi)^n$, expandindo o Binômio de Newton, ou coisas do tipo $\sqrt[n]{z} = \sqrt[n]{a + bi}$, sugiro que use sempre a forma trigonométrica para calculá-las.

\item $n \in \mathbb{Z}$

\item Potenciação: $ z^n = |z|^n \cdot cis(n\cdot\theta)$

\item Radiciação: $ \sqrt[n]{z} = \sqrt[n]{|z|} \cdot cis( \dfrac{\theta + 2k\pi}{n} ) $
	
	
	\end{itemize}

\end{itemize}



\end{document}