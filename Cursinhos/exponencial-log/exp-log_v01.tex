\documentclass[a4paper, 11pt]{article}
\usepackage[top=2cm, bottom=2cm, left=1.5cm, right=1.5cm]{geometry}
\usepackage[utf8]{inputenc}
\usepackage{amsmath, amsfonts, amssymb}
\usepackage{graphicx}
\usepackage[brazil]{babel}



						\begin{document}


\begin{center}
\textbf{Lista de Funções Exponencial e Logarítmo}
\end{center}



\textbf{Resumo:}
\\



\begin{itemize}

\item Propriedades de potenciação:
\\
$a^{x} \cdot a^{y} = a^{x + y}$
\\
$\dfrac{a^{x} }{ a^{y} } = a^{x - y}$
\\
$(a^x)^y = a^{x \cdot y}$
\\
$( \frac{x}{y} )^n = (\frac{x^n}{y^n})$
\\
$a^{x^y} \neq (a^x)^y$
\\
$a^{-1} = \dfrac{1}{a} \longrightarrow a^{-n} = \dfrac{1}{a^n}$
\\
$a^{ \frac{1}{n} }= \sqrt[n]{a}$

\item Definição de função exponencial: $f(x) = b^{x}$ -- "b" é chamado "base"
\\
Caso geral: $f(x) = a \cdot b^{x} + c$

\item Definição de logaritmo: $b^{x} = y \longleftrightarrow log_{b} \ y = x$

\item Função logarítmica: $f(x) = log_{b} \ x$

\item Propriedades de logaritmo:
\\
$b^{x} \longrightarrow log_{b} \ b^{x} = x$
\\
$log_{b} \ (x \cdot y) = log_{b} \ x \, + \, log_{b} \ y$
\\
$log_{b} \ \frac{x}{y} = log_{b} \ x \, - \, log_{b} \ y$
\\
$log_{b} \ x^n = n \cdot log_{b} \ x$
\\
$log_{y} \ x = \dfrac{ log_{b} \ x }{ log_{b} \ y }$ (conversão de base de log)

\end{itemize}

\pagebreak



\begin{center}
\textbf{\underline{EXERCÍCIOS}}
\end{center}

\textbf{\, \ Função exponencial}

\begin{enumerate}

\item Sem usar calculadora, determine o valor das funções abaixo nos pontos indicados. Determine também os domínios de cada uma e suas respectivas imagens.

		\begin{enumerate}
\item $f(x) = 3^x$; $\, \,$ f(0), f(-1), f(1), f(0,5), f(2).
\item $f(x) = {3}^{-x}$; $\, \,$ f(0), f(-1), f(1), f(0,5), f(2).
\item $f(x) = \left( \frac{1}{3} \right)^x$; $\, \,$ f(0), f(-1), f(1), f(0,5), f(2).
\item $f(x) = \frac{1}{2} \cdot 2^x$; $\, \,$ f(0), f(-1), f(1), f(2), f(3).
\item $f(x) = 2^{x-1}$; $\, \,$ f(0), f(-1), f(1), f(2), f(3).
\item $f(x) = 2^{x-3} + \frac{1}{2}$ ; $\, \,$ f(0), f(-1), f(6).
\item $f(x) = 5^{-x}$; $\, \,$ f(-2), f(-0,5), f(3).
\item $f(x) = \left( \frac{1}{4} \right)^{-x}$; $\, \,$ f(0), f(-2), f(1), f(0,5), f(2).
		\end{enumerate}

\item Você notou alguma semelhança nos valores encontrados nos itens (b) e (c) da questão anterior? Explique o que ocorre. Faça o mesmo com os itens (d) e (e) da questão.
\\
\\
\\

\textbf{Gráficos da função exponencial}

\item Em um mesmo plano cartesiano, esboce o gráfico das funções dos itens (a), (b) e (d) da questão 5. Essas funções são crescentes ou decrescentes? Qual delas cresce/decresce mais rapidamente? Quais suas assíntotas? Quais as raízes dessas funções?

\item Relacione o gráfico à função. 
\\
( )$f(x) = 3^x + 1 \, \, \,$ ( )$f(x) = 4^{x – 1} \, \, \,$ ( )$f(x) = 4^{–x} \, \, \,$ ( )$f(x) = 2^x$
\textbf{VOLTAR AQUI}

\item Esboce os gráficos de

	\begin{enumerate}
\item $f(x) = e^x$
\item $g(x) = e^{x – 2}$
\item $h(x) = e^{–x}$
\\
\\
	\end{enumerate}

\textbf{Logaritmos}

\item Usando as leis dos logaritmos, expanda as expressões abaixo.
	\begin{enumerate}
\item $log(4x)$ 
\item $log_2(16x^3)$ 
\item $log_3(y \cdot x^3)$ 
\item $log_2 (\sqrt{x \cdot y})$
\item $log_2 \left( \dfrac{8}{x^2} \right)$
\item $log_2(\dfrac{x}{ w^5 \cdot z^2})$
\item $log_5 \left( \dfrac{x+2}{x^2+1} \right)$
\item $log_3 \left( x\sqrt{x} \right)$
\item $log_3 [\sqrt[3]{x^2 \cdot w}]$
\item $ln \left( \sqrt[3]{ \dfrac{y}{w^4} } \right)$
\item $log_2 \sqrt{x(x+1)}$
	\end{enumerate}

\item Usando as leis dos logaritmos, condense as expressões abaixo.
	\begin{enumerate}
\item $log \, 6 + log \, 5$
\item $log_2 \, x - log_2 \, y$
\item $3 \cdot log_2 \, x + 2 \cdot log_2 \, 5$
\item $\dfrac{log_2 \, x - 3 \cdot log_2 \, z}{2}$
\item $\frac{1}{2} \cdot log_2 \, x + 2 \cdot log_2 \, y - \frac{1}{3} \cdot log_2 \, (x+1)$
\item $\frac{4}{3} \, log_2 \, (x-1) - \frac{1}{3} \, log_2 \, (x + 1)$
\item $-2 \, log_4 \, x$
\item $\frac{1}{3} \, log_2 \, x$
	\end{enumerate}

\item Escreva cada expressão abaixo como o logaritmo de um único termo.
	\begin{enumerate}
\item $\frac{1}{2} \, log_5 \, (x – 1) + log_5 \, (x + 1)$
\item $3 \, log_4 \, (2x + 3) - log_2 \, (x + 2) $
\item $2 \cdot [log  \, (x + 3) \, – \, log \, ( \frac{x}{2} )] \, – \, \frac{3}{2} \, log(x)$
	\end{enumerate}

\item Mostre, com um exemplo, que
	\begin{enumerate}
\item $log \, (a + b) \neq log \, (a) + log \, (b)$
\item $log \, (a \cdot b) \neq log \, (a) · log \, (b)$
	\end{enumerate}


\item Use uma calculadora científica e a regra de mudança de base para aproximar
	\begin{enumerate}
\item $log_2 \, 3$ 
\item $log_5 \, 2$
\item $log_8 \, 24$ 
\item $log_6 \, \frac{1}{12}$
	\end{enumerate}





\end{enumerate}











						\end{document}
